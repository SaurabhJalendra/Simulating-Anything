% Simulating Anything: Domain-Agnostic Scientific Discovery
% via Integrated World Models and Symbolic Regression
%
% Target: AI4Science workshops at NeurIPS/ICML/ICLR
% Format: Workshop paper (4-6 pages + references)

\documentclass{article}

% Use workshop format (adjust for specific venue)
\usepackage[utf8]{inputenc}
\usepackage[T1]{fontenc}
\usepackage{hyperref}
\usepackage{url}
\usepackage{booktabs}
\usepackage{amsmath,amssymb}
\usepackage{graphicx}
\usepackage{microtype}
\usepackage{xcolor}
\usepackage{algorithm}
\usepackage{algorithmic}
\usepackage{subcaption}
\usepackage{multirow}

% Custom commands
\newcommand{\ours}{\textsc{SimAnything}}
\newcommand{\pysr}{\textsc{PySR}}
\newcommand{\sindy}{\textsc{SINDy}}
\newcommand{\rssm}{\textsc{RSSM}}

\title{Simulating Anything: Domain-Agnostic Scientific Discovery \\
via Integrated World Models and Symbolic Regression}

\author{
Saurabh Jalendra \\
\texttt{saurabh@users.noreply.github.com} \\
\url{https://github.com/SaurabhJalendra/Simulating-Anything}
}

\date{}

\begin{document}

\maketitle

% ============================================================================
% ABSTRACT
% ============================================================================
\begin{abstract}
We present \ours{}, a domain-agnostic scientific discovery engine that
autonomously rediscovers known physical laws from simulation data across
14 domains spanning 8 mathematical classes. Given only a natural language
description of a phenomenon, the system builds a simulation, trains an RSSM
world model, explores the parameter space via uncertainty-driven search, and
extracts human-interpretable equations using \pysr{} symbolic regression and
\sindy{} sparse identification. We demonstrate successful rediscovery of
governing equations in all 14 domains, achieving $R^2 \geq 0.999$ in 11 of
14 cases, with a mean $R^2$ of 0.970. Our cross-domain analogy engine
detects 17 structural, dimensional, and topological isomorphisms across
domains. The key architectural insight is that only the simulation layer is
domain-specific ($\sim$50--200 lines per domain); everything else---world
model, exploration, analysis, and reporting---operates on generic tensors,
enabling universal scientific discovery.
\end{abstract}

% ============================================================================
% 1. INTRODUCTION
% ============================================================================
\section{Introduction}
\label{sec:intro}

Scientific discovery has historically been domain-specific: fluid dynamicists
develop Reynolds-averaged equations, ecologists derive Lotka-Volterra
models, and neuroscientists identify FitzHugh-Nagumo dynamics. Each domain
requires bespoke mathematical frameworks, simulation tools, and analytical
techniques. We ask: \emph{can a single computational pipeline rediscover
known physics across fundamentally different domains?}

We present \ours{}, a seven-stage pipeline (Figure~\ref{fig:pipeline})
that takes a natural language problem description as input and produces
discovered equations, phase diagrams, and scaling laws as output. The system
combines:
\begin{itemize}
\item \textbf{Domain-agnostic simulation:} A common \texttt{SimulationEnvironment}
  interface where only the dynamics function is domain-specific.
\item \textbf{World model training:} An RSSM~\cite{hafner2020dreamerv2}
  with 1536 latent dimensions that learns compressed representations of
  any dynamical system.
\item \textbf{Uncertainty-driven exploration:} Monte Carlo dropout uncertainty
  estimates guide parameter space exploration.
\item \textbf{Symbolic regression and sparse identification:} \pysr{}~\cite{cranmer2023pysr}
  and \sindy{}~\cite{brunton2016sindy} extract interpretable equations
  from data.
\end{itemize}

Our key contributions are:
\begin{enumerate}
\item A domain-agnostic architecture where adding a new domain requires
  only $\sim$50--200 lines of simulation code (Section~\ref{sec:architecture}).
\item Successful autonomous rediscovery of governing equations in 14 domains
  across 8 mathematical classes (Section~\ref{sec:experiments}).
\item A cross-domain analogy engine that detects 17 mathematical isomorphisms
  between domains (Section~\ref{sec:cross_domain}).
\item Open-source implementation with 274 tests, 18 publication-quality
  figures, and comprehensive benchmarks.
\end{enumerate}

% ============================================================================
% 2. RELATED WORK
% ============================================================================
\section{Related Work}
\label{sec:related}

\paragraph{Symbolic Regression.}
\pysr{}~\cite{cranmer2023pysr} uses multi-population evolutionary search
to find interpretable mathematical expressions. AI Feynman~\cite{udrescu2020ai}
uses neural networks to discover symmetries before symbolic regression.
Both operate on pre-collected datasets; our system generates its own data
through simulation and exploration.

\paragraph{Sparse Identification of Dynamical Systems.}
\sindy{}~\cite{brunton2016sindy} identifies governing ODEs from time-series
data using sparse regression on a library of candidate functions.
Extensions include PDE-FIND~\cite{rudy2017pdefind} and weak-form
SINDy~\cite{messenger2021weak}. These methods require trajectory data
as input; our pipeline generates and curates this data autonomously.

\paragraph{World Models for Control.}
DreamerV3~\cite{hafner2023dreamerv3} trains RSSM world models for
reinforcement learning policies. We repurpose the same architecture
for scientific \emph{discovery} rather than control: the world model
enables efficient parameter space exploration without re-running
expensive simulations.

\paragraph{AI for Scientific Discovery.}
ScienceWorld~\cite{wang2022scienceworld} evaluates agents on procedural
science tasks. DARWIN~\cite{lu2024darwin} generates research hypotheses.
Neither includes simulation, world model training, or equation discovery.
Our system is the first to integrate all four stages end-to-end.

% ============================================================================
% 3. ARCHITECTURE
% ============================================================================
\section{Architecture}
\label{sec:architecture}

\begin{figure}[t]
\centering
\includegraphics[width=\linewidth]{figures/pipeline.pdf}
\caption{The \ours{} seven-stage pipeline. Gray boxes indicate the only
domain-specific component (simulation dynamics). All other stages operate
on generic tensors.}
\label{fig:pipeline}
\end{figure}

\ours{} consists of seven sequential stages:

\begin{enumerate}
\item \textbf{Problem Architect:} Parses natural language into a structured
  \texttt{ProblemSpec} (variables, parameters, objectives).
\item \textbf{Domain Classifier:} Maps the problem to one of 14 simulation
  domains using rule-based matching with LLM fallback.
\item \textbf{Simulation Builder:} Configures simulation parameters
  (timestep, grid size, initial conditions) via an LLM agent.
\item \textbf{Ground-Truth Simulation:} Runs the domain-specific simulator
  to generate trajectory data. Implements the
  \texttt{SimulationEnvironment} ABC with \texttt{reset()}, \texttt{step()},
  \texttt{observe()}, and \texttt{run()} methods.
\item \textbf{Exploration:} Uses RSSM-based uncertainty estimates to select
  informative parameter configurations via Thompson sampling.
\item \textbf{Analysis:} Applies \pysr{} and \sindy{} to discover equations.
  Includes ablation studies and verification checks.
\item \textbf{Communication:} An LLM agent generates a human-readable
  discovery report in Markdown.
\end{enumerate}

\paragraph{Universality through abstraction.}
The simulation interface requires only four methods (\texttt{reset},
\texttt{step}, \texttt{observe}, \texttt{run}). All upstream and downstream
stages operate on numpy arrays of arbitrary shape. Adding a new domain
means implementing one Python class with $\sim$50--200 lines of dynamics
code. We demonstrate this by adding 14 domains (Table~\ref{tab:domains})
spanning algebraic equations, linear/nonlinear/chaotic ODEs, PDEs, collective
dynamics, and discrete maps.

\paragraph{World model.}
We use an RSSM~\cite{hafner2020dreamerv2} with 512 GRU deterministic units
and $32 \times 32$ categorical stochastic units (1536 total latent dimensions).
The encoder uses CNNs for spatial domains and MLPs for vector-valued
observations. The decoder uses symmetric architectures with symlog output.

% ============================================================================
% 4. EXPERIMENTS
% ============================================================================
\section{Experiments}
\label{sec:experiments}

We evaluate \ours{} on 14 domains across 8 mathematical classes. For each
domain, the system autonomously generates simulation data, identifies the
governing equations via \pysr{} and/or \sindy{}, and compares the discovered
expressions against known analytical results.

\begin{table*}[t]
\centering
\caption{Fourteen-domain rediscovery results. R$^2$ values from PySR symbolic regression and SINDy sparse identification. 11 of 14 domains achieve R$^2 \geq 0.999$.}
\label{tab:results}
\small
\begin{tabular}{rlllcl}
\toprule
\# & Domain & Math Class & Method & R$^2$ & Key Discovery \\
\midrule
1 & Projectile & Algebraic & PySR & \textbf{1.0000} & $0.1019 \approx 1/g$ \\
2 & Lotka-Volterra & Nonlinear ODE & SINDy & \textbf{1.0000} & Exact coefficients \\
3 & Gray-Scott & PDE & PySR & 0.9851 & Wavelength scaling \\
4 & SIR Epidemic & Nonlinear ODE & PySR+SINDy & \textbf{1.0000} & Threshold + ODE \\
5 & Double Pendulum & Chaotic ODE & PySR & \textbf{1.0000} & $4.03 \approx 4\pi^2/g$ \\
6 & Harmonic Osc. & Linear ODE & PySR+SINDy & \textbf{1.0000} & Frequency + ODE \\
7 & Lorenz & Chaotic ODE & SINDy & \textbf{1.0000} & All 3 equations \\
8 & Navier-Stokes & PDE & PySR & \textbf{1.0000} & $4=2|k|^2$ for $(1,1)$ \\
9 & Van der Pol & Nonlinear ODE & PySR & \textbf{1.0000} & $A=2.01$ \\
10 & Kuramoto & Collective & PySR & 0.9695 & Sync order param. \\
11 & Brusselator & Nonlinear ODE & PySR+SINDy & \textbf{0.9999} & Hopf + ODE \\
12 & FitzHugh-Nagumo & Nonlinear ODE & SINDy & \textbf{1.0000} & Exact coefficients \\
13 & Heat Eq. 1D & Linear PDE & PySR & \textbf{1.0000} & Spectral exact \\
14 & Logistic Map & Discrete & PySR & 0.6287 & Fractal spectrum \\
\midrule
\multicolumn{6}{l}{Mean R$^2$ = 0.970 | 11/14 domains with R$^2 \geq 0.999$} \\
\midrule
\multicolumn{6}{l}{\textit{Extended domains (simulation + tests only, same pipeline infrastructure):}} \\
\midrule
15 & Duffing Osc. & Chaotic ODE & -- & -- & Period-doubling chaos \\
16 & Schwarzschild & GR Geodesic & -- & -- & ISCO $= 6M$ \\
17 & Quantum Osc. & Quantum PDE & -- & -- & $E_n = (n+\tfrac{1}{2})\hbar\omega$ \\
18 & Boltzmann Gas & Stat.\ Mech. & -- & -- & $PV = NkT$ \\
19 & Spring Chain & Coupled ODE & -- & -- & $\omega(k)$ dispersion \\
20 & Kepler Orbit & Celestial & -- & -- & $T^2 \propto a^3$ \\
21 & Driven Pend. & Chaotic ODE & -- & -- & Feigenbaum cascade \\
22 & Coupled Osc. & Coupled ODE & -- & -- & Beat frequency \\
23 & Diffusive LV & RD-PDE & -- & -- & Fisher-KPP waves \\
24 & Damped Wave & Wave PDE & -- & -- & $\omega_k = ck$ dispersion \\
25 & 2D Ising & Stat.\ Mech. & -- & -- & $T_c = 2J/\ln(1{+}\sqrt{2})$ \\
\bottomrule
\end{tabular}
\end{table*}

\subsection{Algebraic Equations}
\textbf{Projectile motion.} \pysr{} recovered $R = v_0^2 \cdot 0.1019
\cdot \sin(2\theta)$ from 225 data points (15 speeds $\times$ 15 angles),
with $R^2 = 1.0000$. The coefficient $0.1019$ matches $1/g = 1/9.81 =
0.10194$ to 4 significant figures.

\subsection{Linear ODEs}
\textbf{Harmonic oscillator.} \pysr{} recovered $\omega_0 = \sqrt{k/m}$
($R^2 = 1.0$) and the damping rate $c/(2m)$ ($R^2 = 1.0$). \sindy{}
recovered the exact ODE $\ddot{x} = -4x - 0.4\dot{x}$ (true: $k=4$,
$c=0.4$).

\subsection{Nonlinear ODEs}
\textbf{Lotka-Volterra.} \sindy{} recovered exact ODE coefficients
($R^2 = 1.0$): $\dot{x} = 1.10x - 0.40xy$ (true: $\alpha=1.1$, $\beta=0.4$).
\textbf{SIR epidemic.} \pysr{} found $R_0 = \beta/\gamma$ ($R^2 = 1.0$).
\textbf{Van der Pol.} Period scaling $T(\mu)$ recovered with $R^2 = 0.99996$;
amplitude $A = 2.01$ (theory: $A = 2$).
\textbf{Brusselator.} Hopf threshold $b_c \approx a^2 + 0.91$ ($R^2 = 0.996$;
theory: $b_c = 1 + a^2$). \sindy{} ODE recovery $R^2 = 0.9999$.
\textbf{FitzHugh-Nagumo.} \sindy{} recovered both ODEs with
$R^2 = 0.99999999$, matching all 6 coefficients exactly.

\subsection{Chaotic ODEs}
\textbf{Double pendulum.} \pysr{} found $T = \sqrt{4.03 \cdot L}$ where
$4.03 \approx 4\pi^2/g$ ($R^2 = 0.999993$). Energy conservation verified
to $10^{-7}$.
\textbf{Lorenz attractor.} \sindy{} recovered all three Lorenz equations
($R^2 = 0.99999$) with parameters $\sigma = 9.98$ (true: 10), $\rho = 27.8$
(true: 28), $\beta = 2.66$ (true: $8/3 = 2.667$).

\subsection{PDEs}
\textbf{Gray-Scott.} Wavelength scaling $\lambda \sim \sqrt{D_v}$ recovered
with correlation 0.927 and \pysr{} $R^2 = 0.985$.
\textbf{Navier-Stokes 2D.} Decay rate $\lambda = 4\nu$ ($R^2 = 1.0$),
matching the analytical $2\nu|k|^2$ for Taylor-Green vortex mode $(1,1)$
where $|k|^2 = 2$.
\textbf{Heat equation.} Mode decay rate $\lambda = D$ ($R^2 = 1.0$)
for $k=1$ on $[0, 2\pi]$, exact to machine precision.

\subsection{Collective and Discrete Dynamics}
\textbf{Kuramoto.} Synchronization transition $r(K)$ captured with
$R^2 = 0.97$; critical coupling $K_c = 1.10$ (theory: $4/\pi \approx 1.27$,
14\% finite-size error at $N=100$).
\textbf{Logistic map.} Feigenbaum $\delta$ estimated at $[4.0, 4.75]$
(theory: 4.669). Chaos onset $r_c = 3.576$ (theory: 3.5699). Maximum
Lyapunov exponent $\lambda(r{=}4) = \ln 4 = 1.386$ (exact).

% ============================================================================
% 5. CROSS-DOMAIN ANALYSIS
% ============================================================================
\section{Cross-Domain Analysis}
\label{sec:cross_domain}

\begin{figure}[t]
\centering
\includegraphics[width=0.8\linewidth]{figures/cross_domain_matrix_14domain.pdf}
\caption{Cross-domain analogy strength matrix. 17 isomorphisms detected
across 14 domains. Brighter cells indicate stronger mathematical analogies.}
\label{fig:cross_domain}
\end{figure}

Our cross-domain analogy engine computes three types of similarity between
domain pairs:

\paragraph{Structural analogies} detect shared mathematical structures:
bilinear interaction terms (LV $\leftrightarrow$ SIR), cubic nonlinearities
(Brusselator $\leftrightarrow$ VdP $\leftrightarrow$ FHN), and diffusion
operators (Heat $\leftrightarrow$ NS).

\paragraph{Dimensional analogies} identify parameter correspondences:
$\sqrt{k/m}$ (oscillator) $\leftrightarrow$ $\sqrt{g/L}$ (pendulum)
share the same dimensional structure despite different physical origins.

\paragraph{Topological analogies} detect shared phase space topology:
limit cycles (VdP $\leftrightarrow$ Brusselator), strange attractors
(Lorenz $\leftrightarrow$ Logistic), and synchronization transitions
(Kuramoto $\leftrightarrow$ SIR).

Figure~\ref{fig:cross_domain} shows the $14 \times 14$ analogy matrix
with 17 detected isomorphisms. The strongest analogies (strength $> 0.9$)
connect: LV $\leftrightarrow$ SIR (identical ODE structure),
Pendulum $\leftrightarrow$ Oscillator (shared harmonic restoring force),
and VdP $\leftrightarrow$ Oscillator (limit cycle $\leftrightarrow$
center dynamics).

% ============================================================================
% 6. DISCUSSION
% ============================================================================
\section{Discussion}
\label{sec:discussion}

\paragraph{Why does this work?}
The key insight is that most scientific laws are \emph{low-complexity}
symbolic expressions over \emph{high-dimensional} parameter sweeps.
Systematic exploration of the parameter space generates diverse training
data that constrains \pysr{}'s evolutionary search. The world model
enables efficient exploration without re-running expensive simulations.

\paragraph{Limitations.}
(1) The logistic map's Lyapunov exponent ($R^2 = 0.63$) shows that
fractal quantities resist symbolic fitting---this is a fundamental
limitation of polynomial/analytic function spaces.
(2) Kuramoto's 14\% error in $K_c$ reflects finite-size effects at
$N=100$; larger ensembles improve accuracy but increase compute.
(3) Gray-Scott's moderate $R^2 = 0.985$ reflects the challenge of
fitting PDE scaling laws from limited spatial resolution data.

\paragraph{Scaling.}
Adding a new domain requires only implementing a \texttt{SimulationEnvironment}
subclass ($\sim$50--200 lines). All other pipeline stages operate unchanged.
We have verified this with 14 domains across 8 mathematical classes.

% ============================================================================
% 7. CONCLUSION
% ============================================================================
\section{Conclusion}
\label{sec:conclusion}

\ours{} demonstrates that a single computational pipeline can autonomously
rediscover known physical laws across 14 fundamentally different domains.
The system achieves $R^2 \geq 0.999$ in 11 of 14 domains, with a mean
$R^2$ of 0.970 across all domains. Cross-domain analogy detection reveals
17 mathematical isomorphisms, suggesting that scientific domains share
more structure than is commonly appreciated. Our architecture's key
contribution is the separation of domain-specific simulation from
domain-agnostic discovery, enabling universal scientific inference with
minimal per-domain effort.

% ============================================================================
% REFERENCES
% ============================================================================
\bibliographystyle{plain}
\begin{thebibliography}{10}

\bibitem{brunton2016sindy}
S.~L. Brunton, J.~L. Proctor, and J.~N. Kutz.
\newblock Discovering governing equations from data by sparse identification of
  nonlinear dynamical systems.
\newblock \emph{Proceedings of the National Academy of Sciences}, 113(15):3932--3937, 2016.

\bibitem{cranmer2023pysr}
M.~Cranmer.
\newblock Interpretable machine learning for science with PySR and SymbolicRegression.jl.
\newblock \emph{arXiv preprint arXiv:2305.01582}, 2023.

\bibitem{hafner2020dreamerv2}
D.~Hafner, T.~Lillicrap, M.~Norouzi, and J.~Ba.
\newblock Mastering Atari with discrete world models.
\newblock \emph{arXiv preprint arXiv:2010.02193}, 2020.

\bibitem{hafner2023dreamerv3}
D.~Hafner, J.~Pasukonis, J.~Ba, and T.~Lillicrap.
\newblock Mastering diverse domains through world models.
\newblock \emph{arXiv preprint arXiv:2301.04104}, 2023.

\bibitem{lu2024darwin}
C.~Lu et al.
\newblock The AI Scientist: Towards fully automated open-ended scientific discovery.
\newblock \emph{arXiv preprint arXiv:2408.06292}, 2024.

\bibitem{messenger2021weak}
D.~A. Messenger and D.~M. Bortz.
\newblock Weak SINDy: Galerkin-based data-driven model selection.
\newblock \emph{Multiscale Modeling \& Simulation}, 19(3):1474--1497, 2021.

\bibitem{rudy2017pdefind}
S.~H. Rudy, S.~L. Brunton, J.~L. Proctor, and J.~N. Kutz.
\newblock Data-driven discovery of partial differential equations.
\newblock \emph{Science Advances}, 3(4):e1602614, 2017.

\bibitem{udrescu2020ai}
S.-M. Udrescu and M.~Tegmark.
\newblock AI Feynman: A physics-inspired method for symbolic regression.
\newblock \emph{Science Advances}, 6(16):eaay2631, 2020.

\bibitem{wang2022scienceworld}
R.~Wang et al.
\newblock ScienceWorld: Is your agent smarter than a 5th grader?
\newblock \emph{arXiv preprint arXiv:2203.07540}, 2022.

\end{thebibliography}

\end{document}
